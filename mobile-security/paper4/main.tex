%% bare_conf_compsoc.tex
%% V1.4a
%% 2014/09/17
%% by Michael Shell
%% See:
%% http://www.michaelshell.org/
%% for current contact information.
%%
%% This is a skeleton file demonstrating the use of IEEEtran.cls
%% (requires IEEEtran.cls version 1.8s or later) with an IEEE Computer
%% Society conference paper.
%%
%% Support sites:
%% http://www.michaelshell.org/tex/ieeetran/
%% http://www.ctan.org/tex-archive/macros/latex/contrib/IEEEtran/
%% and
%% http://www.ieee.org/

%%*************************************************************************
%% Legal Notice:
%% This code is offered as-is without any warranty either expressed or
%% implied; without even the implied warranty of MERCHANTABILITY or
%% FITNESS FOR A PARTICULAR PURPOSE! 
%% User assumes all risk.
%% In no event shall IEEE or any contributor to this code be liable for
%% any damages or losses, including, but not limited to, incidental,
%% consequential, or any other damages, resulting from the use or misuse
%% of any information contained here.
%%
%% All comments are the opinions of their respective authors and are not
%% necessarily endorsed by the IEEE.
%%
%% This work is distributed under the LaTeX Project Public License (LPPL)
%% ( http://www.latex-project.org/ ) version 1.3, and may be freely used,
%% distributed and modified. A copy of the LPPL, version 1.3, is included
%% in the base LaTeX documentation of all distributions of LaTeX released
%% 2003/12/01 or later.
%% Retain all contribution notices and credits.
%% ** Modified files should be clearly indicated as such, including  **
%% ** renaming them and changing author support contact information. **
%%
%% File list of work: IEEEtran.cls, IEEEtran_HOWTO.pdf, bare_adv.tex,
%%                    bare_conf.tex, bare_jrnl.tex, bare_conf_compsoc.tex,
%%                    bare_jrnl_compsoc.tex, bare_jrnl_transmag.tex
%%*************************************************************************


% *** Authors should verify (and, if needed, correct) their LaTeX system  ***
% *** with the testflow diagnostic prior to trusting their LaTeX platform ***
% *** with production work. IEEE's font choices and paper sizes can       ***
% *** trigger bugs that do not appear when using other class files.       ***                          ***
% The testflow support page is at:
% http://www.michaelshell.org/tex/testflow/



\documentclass[conference,compsoc]{IEEEtran}
% Some/most Computer Society conferences require the compsoc mode option,
% but others may want the standard conference format.
%
% If IEEEtran.cls has not been installed into the LaTeX system files,
% manually specify the path to it like:
% \documentclass[conference,compsoc]{../sty/IEEEtran}





% Some very useful LaTeX packages include:
% (uncomment the ones you want to load)


% *** MISC UTILITY PACKAGES ***
%
%\usepackage{ifpdf}
% Heiko Oberdiek's ifpdf.sty is very useful if you need conditional
% compilation based on whether the output is pdf or dvi.
% usage:
% \ifpdf
%   % pdf code
% \else
%   % dvi code
% \fi
% The latest version of ifpdf.sty can be obtained from:
% http://www.ctan.org/tex-archive/macros/latex/contrib/oberdiek/
% Also, note that IEEEtran.cls V1.7 and later provides a builtin
% \ifCLASSINFOpdf conditional that works the same way.
% When switching from latex to pdflatex and vice-versa, the compiler may
% have to be run twice to clear warning/error messages.






% *** CITATION PACKAGES ***
%
\ifCLASSOPTIONcompsoc
  % IEEE Computer Society needs nocompress option
  % requires .sty v4.0 or later (November 2003)
  \usepackage[nocompress]{cite}
\else
  % normal IEEE
  \usepackage{cite}
\fi
% cite.sty was written by Donald Arseneau
% V1.6 and later of IEEEtran pre-defines the format of the cite.sty package
% \cite{} output to follow that of IEEE. Loading the cite package will
% result in citation numbers being automatically sorted and properly
% "compressed/ranged". e.g., [1], [9], [2], [7], [5], [6] without using
% cite.sty will become [1], [2], [5]--[7], [9] using cite.sty. cite.sty's
% \cite will automatically add leading space, if needed. Use cite.sty's
% noadjust option (cite.sty V3.8 and later) if you want to turn this off
% such as if a citation ever needs to be enclosed in parenthesis.
% cite.sty is already installed on most LaTeX systems. Be sure and use
% version 5.0 (2009-03-20) and later if using hyperref.sty.
% The latest version can be obtained at:
% http://www.ctan.org/tex-archive/macros/latex/contrib/cite/
% The documentation is contained in the cite.sty file itself.
%
% Note that some packages require special options to format as the Computer
% Society requires. In particular, Computer Society  papers do not use
% compressed citation ranges as is done in typical IEEE papers
% (e.g., [1]-[4]). Instead, they list every citation separately in order
% (e.g., [1], [2], [3], [4]). To get the latter we need to load the cite
% package with the nocompress option which is supported by cite.sty v4.0
% and later.





% *** GRAPHICS RELATED PACKAGES ***
%
\ifCLASSINFOpdf
\usepackage[pdftex]{graphicx}
  % declare the path(s) where your graphic files are
\graphicspath{{./png}}
  % and their extensions so you won't have to specify these with
  % every instance of \includegraphics
\DeclareGraphicsExtensions{.pdf,.jpeg,.png}
\else
  % or other class option (dvipsone, dvipdf, if not using dvips). graphicx
  % will default to the driver specified in the system graphics.cfg if no
  % driver is specified.
  % \usepackage[dvips]{graphicx}
  % declare the path(s) where your graphic files are
  % \graphicspath{{../eps/}}
  % and their extensions so you won't have to specify these with
  % every instance of \includegraphics
  % \DeclareGraphicsExtensions{.eps}
\fi
% graphicx was written by David Carlisle and Sebastian Rahtz. It is
% required if you want graphics, photos, etc. graphicx.sty is already
% installed on most LaTeX systems. The latest version and documentation
% can be obtained at: 
% http://www.ctan.org/tex-archive/macros/latex/required/graphics/
% Another good source of documentation is "Using Imported Graphics in
% LaTeX2e" by Keith Reckdahl which can be found at:
% http://www.ctan.org/tex-archive/info/epslatex/
%
% latex, and pdflatex in dvi mode, support graphics in encapsulated
% postscript (.eps) format. pdflatex in pdf mode supports graphics
% in .pdf, .jpeg, .png and .mps (metapost) formats. Users should ensure
% that all non-photo figures use a vector format (.eps, .pdf, .mps) and
% not a bitmapped formats (.jpeg, .png). IEEE frowns on bitmapped formats
% which can result in "jaggedy"/blurry rendering of lines and letters as
% well as large increases in file sizes.
%
% You can find documentation about the pdfTeX application at:
% http://www.tug.org/applications/pdftex





% *** MATH PACKAGES ***
%
%\usepackage[cmex10]{amsmath}
% A popular package from the American Mathematical Society that provides
% many useful and powerful commands for dealing with mathematics. If using
% it, be sure to load this package with the cmex10 option to ensure that
% only type 1 fonts will utilized at all point sizes. Without this option,
% it is possible that some math symbols, particularly those within
% footnotes, will be rendered in bitmap form which will result in a
% document that can not be IEEE Xplore compliant!
%
% Also, note that the amsmath package sets \interdisplaylinepenalty to 10000
% thus preventing page breaks from occurring within multiline equations. Use:
%\interdisplaylinepenalty=2500
% after loading amsmath to restore such page breaks as IEEEtran.cls normally
% does. amsmath.sty is already installed on most LaTeX systems. The latest
% version and documentation can be obtained at:
% http://www.ctan.org/tex-archive/macros/latex/required/amslatex/math/





% *** SPECIALIZED LIST PACKAGES ***
%
%\usepackage{algorithmic}
% algorithmic.sty was written by Peter Williams and Rogerio Brito.
% This package provides an algorithmic environment fo describing algorithms.
% You can use the algorithmic environment in-text or within a figure
% environment to provide for a floating algorithm. Do NOT use the algorithm
% floating environment provided by algorithm.sty (by the same authors) or
% algorithm2e.sty (by Christophe Fiorio) as IEEE does not use dedicated
% algorithm float types and packages that provide these will not provide
% correct IEEE style captions. The latest version and documentation of
% algorithmic.sty can be obtained at:
% http://www.ctan.org/tex-archive/macros/latex/contrib/algorithms/
% There is also a support site at:
% http://algorithms.berlios.de/index.html
% Also of interest may be the (relatively newer and more customizable)
% algorithmicx.sty package by Szasz Janos:
% http://www.ctan.org/tex-archive/macros/latex/contrib/algorithmicx/




% *** ALIGNMENT PACKAGES ***
%
%\usepackage{array}
% Frank Mittelbach's and David Carlisle's array.sty patches and improves
% the standard LaTeX2e array and tabular environments to provide better
% appearance and additional user controls. As the default LaTeX2e table
% generation code is lacking to the point of almost being broken with
% respect to the quality of the end results, all users are strongly
% advised to use an enhanced (at the very least that provided by array.sty)
% set of table tools. array.sty is already installed on most systems. The
% latest version and documentation can be obtained at:
% http://www.ctan.org/tex-archive/macros/latex/required/tools/


% IEEEtran contains the IEEEeqnarray family of commands that can be used to
% generate multiline equations as well as matrices, tables, etc., of high
% quality.




% *** SUBFIGURE PACKAGES ***
%\ifCLASSOPTIONcompsoc
%  \usepackage[caption=false,font=footnotesize,labelfont=sf,textfont=sf]{subfig}
%\else
%  \usepackage[caption=false,font=footnotesize]{subfig}
%\fi
% subfig.sty, written by Steven Douglas Cochran, is the modern replacement
% for subfigure.sty, the latter of which is no longer maintained and is
% incompatible with some LaTeX packages including fixltx2e. However,
% subfig.sty requires and automatically loads Axel Sommerfeldt's caption.sty
% which will override IEEEtran.cls' handling of captions and this will result
% in non-IEEE style figure/table captions. To prevent this problem, be sure
% and invoke subfig.sty's "caption=false" package option (available since
% subfig.sty version 1.3, 2005/06/28) as this is will preserve IEEEtran.cls
% handling of captions.
% Note that the Computer Society format requires a sans serif font rather
% than the serif font used in traditional IEEE formatting and thus the need
% to invoke different subfig.sty package options depending on whether
% compsoc mode has been enabled.
%
% The latest version and documentation of subfig.sty can be obtained at:
% http://www.ctan.org/tex-archive/macros/latex/contrib/subfig/




% *** FLOAT PACKAGES ***
%
%\usepackage{fixltx2e}
% fixltx2e, the successor to the earlier fix2col.sty, was written by
% Frank Mittelbach and David Carlisle. This package corrects a few problems
% in the LaTeX2e kernel, the most notable of which is that in current
% LaTeX2e releases, the ordering of single and double column floats is not
% guaranteed to be preserved. Thus, an unpatched LaTeX2e can allow a
% single column figure to be placed prior to an earlier double column
% figure. The latest version and documentation can be found at:
% http://www.ctan.org/tex-archive/macros/latex/base/


%\usepackage{stfloats}
% stfloats.sty was written by Sigitas Tolusis. This package gives LaTeX2e
% the ability to do double column floats at the bottom of the page as well
% as the top. (e.g., "\begin{figure*}[!b]" is not normally possible in
% LaTeX2e). It also provides a command:
%\fnbelowfloat
% to enable the placement of footnotes below bottom floats (the standard
% LaTeX2e kernel puts them above bottom floats). This is an invasive package
% which rewrites many portions of the LaTeX2e float routines. It may not work
% with other packages that modify the LaTeX2e float routines. The latest
% version and documentation can be obtained at:
% http://www.ctan.org/tex-archive/macros/latex/contrib/sttools/
% Do not use the stfloats baselinefloat ability as IEEE does not allow
% \baselineskip to stretch. Authors submitting work to the IEEE should note
% that IEEE rarely uses double column equations and that authors should try
% to avoid such use. Do not be tempted to use the cuted.sty or midfloat.sty
% packages (also by Sigitas Tolusis) as IEEE does not format its papers in
% such ways.
% Do not attempt to use stfloats with fixltx2e as they are incompatible.
% Instead, use Morten Hogholm'a dblfloatfix which combines the features
% of both fixltx2e and stfloats:
%
\usepackage{dblfloatfix}
% The latest version can be found at:
% http://www.ctan.org/tex-archive/macros/latex/contrib/dblfloatfix/




% *** PDF, URL AND HYPERLINK PACKAGES ***
%
%\usepackage{url}
% url.sty was written by Donald Arseneau. It provides better support for
% handling and breaking URLs. url.sty is already installed on most LaTeX
% systems. The latest version and documentation can be obtained at:
% http://www.ctan.org/tex-archive/macros/latex/contrib/url/
% Basically, \url{my_url_here}.


% *** Do not adjust lengths that control margins, column widths, etc. ***
% *** Do not use packages that alter fonts (such as pslatex).         ***
% There should be no need to do such things with IEEEtran.cls V1.6 and later.
% (Unless specifically asked to do so by the journal or conference you plan
% to submit to, of course. )


% correct bad hyphenation here
\hyphenation{op-tical net-works semi-conduc-tor}


%************************************************************************
%************************************************************************%************************************************************************
%************************************************************************


\usepackage{url}
\usepackage{framed}
\usepackage{fancyvrb}
\usepackage{listings}
%\usepackage[usenames, dvipsnames]{color}
\usepackage[all]{nowidow}
\usepackage{solarized-light}
%\lstset{
%	frame=single,
%	basicstyle=\ttfamily,
%	breaklines
%}
\lstdefinestyle{mystyle}{
    basicstyle=\footnotesize,
    breakatwhitespace=false,         
    breaklines=true,                 
    captionpos=b,                    
    keepspaces=true,                 
    numbers=left,                    
    numbersep=5pt,                  
    showspaces=false,                
    showstringspaces=false,
    showtabs=false,                  
    tabsize=2,
}
 
\lstset{style=mystyle}
\usepackage{tabularx}
\usepackage{flushend}


%************************************************************************
%************************************************************************
%************************************************************************
%************************************************************************

\begin{document}
\title{14-829: Mobile Security - Fall 2015 - Assignment 4}
\author{\IEEEauthorblockN{Michael Appel}
\IEEEauthorblockA{Information Networking Institute\\
Carnegie Mellon University\\
Pittsburgh, PA\\
Email: moappel@cmu.edu}}
% make the title area
\maketitle
%************************************************************************


\begin{abstract}
GServices is an account management app that is installed by hundreds of thousands, and was created by a large Philippine telecommunications company, Globe. However, this app makes numerous mistakes in Android coding practices, which make its users vulnerable to a number of attacks. This paper analyzes GServices for insecure data storage, data leakage, incorrect permission models, and the presence of binary protections. 
\end{abstract}



\section{Introduction}
% For Task 1, provide a high level overview of the app you chose. Explain its functionality and why you chose it.
This homework analyzes the Anroid app, GServices \cite{globefeatures}, which is an account management app for a Philippine telecom provider \cite{globe}. The app is available for free on Google Play\cite{gservices}, and has 100,000 to 500,000 installs. There are significant features such as enrolling additional phone numbers on prepaid or postpaid plans, checking account details, and subscribing to call, text, and "surf" promos \cite{globefeatures}.

I chose this app to analyze after reading a paper that examined branchless mobile banking apps \cite{190884}. Reaves et al analyzed another app by Globe Telecom called, GCash, that performed poorly under analysis. I made the assumption that Globe was probably in the practice of making poor apps, and looked for others from the same company.

Unfortunately, the full workings of the app are not exactly known to me. You are required to have an account with Globe Telecom, and the app is first setup with that phone number. I attempted to bypass this by setting the value "is\_logged\_in" to true in the app's sharedPreferences after determining this was the flag to set. Failing that, I found checks for "PIN\_state" values, which indicate whether a PIN has been entered.  However, this also did not succeed. I am not confident I can bypass the initial account setup in a short amount of time so I proceeded with static analysis. Given the complete lack of obfuscations this was relatively straight-forward, but the app was quite large and the coverage of analysis did not extend into the many imported libraries.

This paper is organized as follows: first I return the details of my analysis, and then make recommendations for attacks and hardening. The analysis will follow from binary protections, to data leakage and storage, and finally I cover the Android permission model. The recommendations will cover attacks, and then how to mitigate those and others.

\section{Analysis}
% For Task 2, provide a detailed summary of your app analysis efforts for each of the risks.
This analysis demonstrates the ability to trivially get encryption keys from GServices through a third party app, shows that SSL communications are unprotected, and user data is leaked or can be intercepted and eavesdropped. The overall work can be quickly noted in table \ref{table:t1}.

\begin{table*}[t]
\centering
\begin{tabular}{ |p{.25cm}|p{3cm}|p{3cm}|p{3cm}|p{3cm}| }
\hline
& Risks & Analysis & Vulnerabilities & Mitigation \\
\hline \hline
1 & Lack of Binary Protections &
% Analysis
No obfuscation, no root checks, and no integrity checks. &
% Vuln
Lack of obfuscation allows for easier reverse engineering, which reveals hard coded values and faults in the program.\newline
Adversaries can use rooted devices to extract data previously secured by the Android UID sandbox.\newline
Modification of the code is also not detected at runtime without any kind of verification of integrity.
&
% Mitigation
Re-enable ProGuard's default settings for obfuscation.
Add checks for the ability to gain root, or the presence of observable indicators of rooted devices.
Add runtime code verification, and store checksums off the device. \\
\hline

2 & Unintended Data Leakage
&
% Analysis
Leaks through logging and intent receivers, which lack permissions, and disables SSL protections.
&
% Vuln
Apps log sensitive data to the Android logging system, and adversaries can potentially gain access to this data. The data is not encrypted in the log, and can be achieved through ADB if there is physical access to the device.\newline
Disabling SSL makes this app vulnerable to MITM attacks, which are especially problematic when dealing with account records and personal financial data.
& 
% Mitigation
In order to reduce the amount of data available for eavesdroppers in the logs it is necessary to sanitize the logged input. It is not a good idea to log user account details on a release build of an app.
SSL simply requires a change in verifier options to remove the blanket allow all. Then the name checking will be reinstated.
Intents can be fixed with the mitigations in the permission model analysis.
\\
\hline

3 & Insecure Data Storage &
% Analysis
Uses external storage for personal photos, and stores all data in plain-text.
&
Adversaries do not need any special privilege to read external storage. The personal photos can be retrieved by anyone.\newline
In the event of a compromised device, especially a rooted device, all data stored in plain-text is essentially handed over to the attacker.\newline
It then becomes trivial to read an SQLite database or parse a JSON file to extract user data.
&
Personal photos can be encrypted before writing to disk in order to take advantage of the external storage space.\newline
Likewise, enabling the setStorageEncryption option on the app will make the user data much more difficult to read in the event the device is compromised.
\\
\hline

4 & Incorrect Permission Model &
% Analysis
Fails to constrain BroadCaster receivers with Intent permissions. Does not obey least privilege.
&
Intent receivers without permissions are vulnerable to snooping and injected traffic. This app does not make use of any permission, and passes user data without encryption.\newline
This app's poor coding practices and large amount of permissions make it a target to be misused by other applications, which may try to leak the permissions from GServices.
&
GServices can add permissions to broadcasting intents, and sanitize the input to the logging system.
The developers can also make conscious efforts to reduce the number of permissions needed.
\\
\hline
\end{tabular}
\caption{Overall Vulnerability Analysis Results}
\label{table:t1}
\end{table*}


\subsection{Binary Protections}
The GServices app does not employ any binary protections. It is not obfuscated, there are no checks for evidence of root or debugging, and there are no code integrity checks at runtime. These issues are compounded by the fact that the developers hard coded several secrets into the program as obviously named variables, and stored user data in plain text.

First, the GServices app reveals a hardcoded secret in conjunction with poor coding practices to completely defeat the protections of encryption and transport layer security (TLS/SSL). Analyzing the class, AESEncryptDecrypt, shows that GServices uses the AES algorithm in electronic code book (ECB) mode. ECB mode is usually warned against because it is more vulnerable to cryptanalysis and replay attacks than other modes. ECB lacks an initialization vector (an initial small amount of entropy) at the start, so a key is all that's needed. Also, ECB works on set blocks so attacks using chosen plain text are fairly simple. Replays are also possible by substituting blocks of cipher text. Thus, the presence of cryptography was thought to buy assurance, but the misapplication of cipher modes made the protections weaker than they should have been.

However, none of this cryptanalysis is necessary because GServices gives away the secret key to its encryption function. Listing \ref{lst:secretkey} shows a key string stored as a private variable, but the variable is exposed to the world by a public method, \texttt{getPrivateKey}, which returns the string value. A malicious adversary can now access the key by legal means, which I show to be trivial in listing \ref{lst:getkey}. It is important to note the only knowledge required to immediately understand this secret is a prior knowledge of AES or basic decryption when reading GService's Java class filenames.


\begin{lstlisting}[language=Java, caption=AESEncryptDecrypt class contains secrets that are made publicly available, label={lst:secretkey}]
private static String key = "3CF3443B9A93CBAD987B18BEACCFC";
...
public static String getPrivateKey()
    { return key; }
\end{lstlisting}

\begin{lstlisting}[language=Java, caption=Retrieve secret from GServices app without any permission, label={lst:getkey}]
Context ctx = this.createPackageContext("com.seerlabs.GlobeSSA", Context.CONTEXT_INCLUDE_CODE | Context.CONTEXT_IGNORE_SECURITY);
Class<?> cls = Class.forName("com.seerlabs.GlobeSSA.piwik.AESEncryptDecrypt", true, ctx.getClassLoader());
Method mth = cls.getMethod("getPrivateKey", (Class[]) null);
String key = (String) mth.invoke(null, (Object[])null);
Log.e("CTX", "Got key: " + key);
\end{lstlisting}

Next, evidence of hardcoded secrets can be seen frequently in unobfuscated code as in listing \ref{lst:snips1} and \ref{lst:snips2}. These particular values are not explicitly secrets, but the ease of which an attacker can gain them because of the lack of binary protections presents issues. For example, the variable on line 2, \texttt{NOTIFIER\_NAME}, is used to register the Apigee data client. The data client receives notifications based on these credentials and a server generated device id (which does not need to be provided)\cite{apigee:sdk}. This allows an adversary to trivially obtain the credentials to GServices' cloud accounts. The credentials exposed may not be administrator account logins, but they would allow the spoofing of that account in a maliciously developed app. That malicious app may, for example, be able to snoop on intents or messaging services. 

\begin{lstlisting}[language=Java, caption=MainActivity snippets of hard coded identifiers,label={lst:snips1}]
// Apigee Client Notifier
private static final String NOTIFIER_NAME = "androidDev";
// Amazon Web Services Notification
final AmazonSNSClient amazonSNSClient = new AmazonSNSClient(new CognitoCachingCredentialsProvider((Context)this,
    "us-east-1:d8c0fc62-9055-45da-a47e-b8ef183f6fd6",
    Regions.US_EAST_1));
// Google Cloude Messaging
MainActivity.this.regid = MainActivity.this.gcm.register("568692659701");
\end{lstlisting}

\begin{lstlisting}[language=Java, caption=HttpUtil snippets of sensitive data, label={lst:snips2}]
private static final String DEV_APIKEY = "9b308b01f01121edc9c3a961d7aa2015";
    private static final String DEV_LOGIN_CLIENT_ID = "s6kN595rmvnoo0DjwhkWON0tXAVmL6GB";
    private static final String DEV_LOGIN_CLIENT_SECRET = "0ctcOBaiJ4d5ES5n";
    private static final String DEV_URL = "https://rcdelacruz-prod.apigee.net/v2/greygoose-dev";
    private static final String EMERGENCY_LOGIN_CLIENT_ID = "t69Mlw4fjTQlpTs7OWcGawpVgdvvf6jy";
    private static final String EMERGENCY_LOGIN_CLIENT_SECRET = "XfTVmA5XFX6hIfjr";
    private static final String EMERGENCY_URL = "http://codewarrior-test.apigee.net/hello-world";
    private static final String HEADER_AUTHORIZATION_BEARER = "Bearer ";
    private static final String HEADER_AUTHORIZATION_KEY = "Authorization";
\end{lstlisting}


\subsection{Data Leakage}
This section is concerned with data that leaks from an app because developers were not aware of information flowing through a side channel. GServices leaks data through logging and intent broadcasts, which are common bad practices in Android. However, GServices also completely disables the protections of transport layer security (TLS), and make the app vulnerable to man-in-the-middle (MITM) attacks.

The GServices app creates an SSL socket factory in listing \ref{lst:fuckssl}, which returns secure sockets for the HTTP clients requests, however it uses the  verification option \texttt{ALLOW\_ALL\_HOSTNAME\_VERIFIER}. This option disables name checking \cite{Google:hostname} from certificates to sites, which removes all authentication guarantees from web based communication. This option allows any adversary intercept, decrypt, and snoop or modify on all traffic from the app. This traffic includes sensitive information like account numbers and balances for the users telephone service.

\begin{lstlisting}[language=Java, caption=Insecure name checking mode in SSL options, label={lst:fuckssl}]
final MySSLSocketFactory mySSLSocketFactory = httpUtil.new MySSLSocketFactory(instance);            mySSLSocketFactory.setHostnameVerifier(MySSLSocketFactory.ALLOW_ALL_HOSTNAME_VERIFIER);
            return mySSLSocketFactory;
\end{lstlisting}

Next, the GServices developers make extensive use of logging. This creates a channel in which sensitive information, like that in listing \ref{lst:log}, can be read by programs like logcat. Any app or user with permission can read these logs, however they do require elevated permissions because the READ\_LOGS permission is no longer granted to third parties \cite{Google:readlog}. Adversaries with access to the device can monitor this data through the android debug bridge (ADB). 

\begin{lstlisting}[language=Java, caption=A small sample of GServices leaking data through logs, label={lst:log}]
// Other account details are also logged
Log.v(this.TAG, "sMobileNo is " + sMobileNo + " sPlanType is " + sPlanType);
// User location is logged with exact coordinates
Log.d("MapsV2", "userLocation DETECTED -> " + this.userLocation);
Log.d("MapsV2", "ll Lat -> " + lat);
Log.d("MapsV2", "ll Long -> " + lng);
// This registration ID is returned from AWS
Log.v(TAG, "REGISTRATION ID IS :: " + regId);
\end{lstlisting}

Finally, as noted in the binary protections section, the GServices app is vulnerable to leaking data through the cloud services used for messaging and notification in addition to any analytics involved. GServices is approximately 25 MB, and includes a great deal of packages. Some of which were not known or reversed in the interest of time for this homework. Therefore, it is possible because the app is not completely discovered that there are analytic frameworks and other plugins which have hard coded credentials in the app that make them vulnerable to account spoofing.


\subsection{Data Storage}
GServices is also not completely secure when storing user and application data. Everything is stored in plain text, and no use of \texttt{setStorageEncryption} \cite{Google:storageEncryption} was found. The most common locations are in SharedPreferences, and SQLite databases. However, external storage is also used to store photos. My analysis is not exactly clear on what the photos are, but they are called from ProfileSetup and AccountInformation activities so it may be that they are profile pictures, which are colloquially known as "selfies." Any third party app has access to these files.

The data stored in SharedPreferences can include sensitive information like registration ID's, see listing \ref{lst:broadcast}, and user account details. The SQLite databases use the schema in listing \ref{lst:sql}, and make no attempt to encrypt the data. I used manual analysis to find uses of the uncovered encryption functions, and examined interactions with the AccountMatrix class to determine there were no protections used.

\begin{lstlisting}[language=Java, caption=Database scheme for user accounts, label={lst:sql}]
sqlitedatabase.execSQL("CREATE TABLE IF NOT EXISTS accounts (_id integer primary key autoincrement, number text not null, email text not null, name text not null, photoPath text not null, accountDetails text not null, accountBalance text not null, rewardsPoints text not null, planType text not null, alias text not null);");
\end{lstlisting}


\subsection{Android Permission Model}

Due to the large number of third party libraries, and the robust functionality of GServices the app requires many permissions. AndroidStudio crashes running code inspection on the decompiled source. So, it is not certain if all the permissions are actually used, but it seems plausible given the size of the app and libraries. However, there are several manufacturer specific permissions, which must not be necessary in all builds. My analysis of the app does not seem to indicate any need for something like the \texttt{WAKE\_LOCK} permission either. Thus, it seems like the developers could make efforts to reduce these dependencies.

GServices does not implement permissions on any instances of BroadcastReceivers, see listing \ref{lst:broadcast} for examples. Android provides the ability register receivers dynamically and statically \cite{Google:broadcast}, and GServices uses both approaches. However, the app disregards the permissions in both cases. This opens up the communications from and to the app to third parties\cite{Google:broadcast}.

\begin{lstlisting}[language=Java, caption=BroadcastReceiver permissions implemented insecurely, label={lst:broadcast}]
// Dynamic
registerReceiver(this.terminatorReceiver, new IntentFilter(ACTION_TERMINATE));
// Static, AndroidManifest.xml
<receiver android:name="com.seerlabs.GlobeSSA.GServicesUpdateReceiver">
  <intent-filter>
    <action android:name="android.intent.action.MY_PACKAGE_REPLACED"/>
  </intent-filter>
</receiver>
\end{lstlisting}


\section{Recommendations}
% For Task 3, provide an explanation of your idea to break the app, and how you would fix it; or provide an explanation of how you would make the app even more secure than it is.

\subsection{Attacks}
The GServices app is open to several attacks. My proof of concept shows how easy it would be to get the secret key even from a third party app with no permissions. I believe more opportunities like this exist because of the prevalence of public methods in GServices' classes. If this key and the SSL vulnerability from listing \ref{lst:fuckssl} are combined then we can decode all app traffic to the Globe servers. The SSL vulnerability allows a MITM attack, but the parameters are base64 encoded and encrypted. Fortunately, we have the key! This attack is only limited by the number of devices we can MITM. For example, establishing a wireless access point in a crowded area could yield a great deal of data.

If the key is removed the attacker can still eavesdrop, and attempt to inject intents into the BroadcastReceivers of GServices. At the very least, the attacker can cause a denial-of-service by repeatedly sending Intents constructed with \texttt{MainActivity.TERMINATE}, which triggers the GServices cleanup methods.

Finally, adversaries can easily reverse engineer the app in its current state. This allows access to all the credentials and ids in listing \ref{lst:snips1} and \ref{lst:snips2}. I did not provide a proof of concept of this attack, but from the API's it seems possible to duplicate the app's registration, or connect to developer accounts. In the first case attackers can register duplicate listeners to receive app push notifications and modify analytics. In the second, adversaries may be able to gain elevated privileges in the company's cloud account.

\subsection{Hardening}
In order to harden the app against reverse engineering it needs to employ more binary protections. This includes obfuscation of source code, which can be accomplished by leaving the default setting of Android's ProGuard enabled in Android Studio. This should help to secure hard coded cloud service account credentials from easy duplication and mimicry. Additionally, implementing key functions in lower-level languages like C/C++ make it more difficult to reverse engineer their functionality.

However, adversaries will always be able to reverse an application, especially in Java, given the time and incentive. Therefore, it is important to offload as much as possible that developers want to keep secret. This can include credentials, algorithms, or user data.

Next, in the launch \texttt{SplashScreenActivity}, the app should implement checks for root. There are numerous, documented methods to do this \cite{owaspbinary}. This counter-measure is especially important here because of all the poor data storage. With everything in plain text malware or other users at elevated privileges on rooted phones have easy access to plain-text user data. 

Finally, the app should implement the permissions in static and dynamic broadcast receivers. Some user data is transmitted through these channels, and both receive and transmit ends of the channels are open to forged messages and eaves dropping.


%************************************************************************
\clearpage
\newpage
\bibliography{bilbo.bib}{}
\bibliographystyle{IEEEtran}
\end{document}